% !TeX root = RJwrapper.tex
\title{Draft updates of Eastern Bering Sea Pollock assessment}
\author{by James Ianelli, Author Two}

\maketitle

\abstract{%
An abstract of less than 150 words.
}


\hypertarget{summary-of-pollock-results}{%
\subsection{Summary of pollock
results}\label{summary-of-pollock-results}}

\begin{table}[ht]
\centering
\begin{tabular}{lrrrr}
  \hline
       & \multicolumn{2}{c}{As estimated or $\mathit{specified}$ } & \multicolumn{2}{c}{As estimated or $\mathit{recommended}$ }  \\
       & \multicolumn{2}{c}{specified $\mathit{last}$ year for:}  & \multicolumn{2}{c}{$\mathit{this}$ year for: }               \\
        Quantity & 2017      &2018   & 2018      &2019 \\ 
  \hline
    M (natural mortality rate, ages 3+) &   0.3 &   0.3 &   0.3 &   0.3 \\
Tier    &   1a  &   1a  &   1a  &   1a \\
Projected total (age 3+) biomass (t)    &   13,000,000 t    &   12,100,000 t    &       &    \\
Projected female spawning biomass (t) & 4,600,000 t &   4,500,000 t &       &    \\
$B_0$                                 & 5,700,000 t &   5,700,000 t &       &    \\
$B_{msy}$                             & 2,165,000 t &   2,165,000 t &       &    \\
$F_{OFL}$                             & 0.465   &   0.465   &       &    \\
$maxF_{ABC}$                        &   0.398   &   0.398   &       &    \\
$F_{ABC}$ & 0.36    &   0.37    &       &    \\
$OFL$ (t)   &   3,640,000 t &   4,360,000 t &       &    \\
$maxABC$ (t)    &   3,120,000 t &   3,740,000 t &       &    \\
$ABC$ (t)   &   2,800,000 t &   2,979,000 t &       &    \\
Status  &   2015    &   2016    &       &    \\
Overfishing &   No  &   n/a &   No  &   n/a \\
Overfished  &   n/a &   No  &   n/a &   No \\
Approaching overfished  &   n/a &   No  &   n/a &   No \\
\hline
\end{tabular}
\end{table}

\hypertarget{data}{%
\section{Data}\label{data}}

New data presented in this assessment suggests that the above average
2008 year-class is slightly higher than before and that the 2012
year-class also appears to be above average. As such, the maximum
permissible Tier~1a ABC remains high. Tier 3 estimates of ABC are also
quite high; however, besides adding stability in catch rates and effort,
an ABC based on the Tier 3 values is recommended (2,800,000 t) which is
well below the maximum permissible (Tier 1a) value of 3,120,000 t. The
Tier 1a overfishing level (OFL) is estimated to be 3,640,000 t.

\hypertarget{response-to-ssc-and-plan-team-comments}{%
\subsection{Response to SSC and Plan Team
comments}\label{response-to-ssc-and-plan-team-comments}}

\hypertarget{general-comments}{%
\subsubsection{General comments}\label{general-comments}}

From the December 2015 SSC minutes: The SSC reminds the authors and PTs
to follow the model-numbering scheme adopted at the December 2014
meeting.

We followed the model-numbering scheme described in the most recent
version of the SAFE Guidelines (Option D). The SSC encourages the
authors and PTs to refer to the forthcoming CAPAM data-weighting
workshop report. Sample sizes for the fishery data were re-evaluated to
obtain alternative time-varying inputs---these were rescaled according
to estimated ``Francis weights'' (method TA1.8; Francis 2011) from model
fits and evaluated against alternative levels of flexibility in time and
age-varying selectivity.

The SSC recommends that assessment authors work with AFSC's survey
program scientist to develop some objective criteria to inform the best
approaches for calculating Q with respect to information provided by
previous survey trawl performance studies (e.g.~Somerton and Munro
2001), and fish-temperature relationships which may impact Q. The survey
catchability was freely estimated in this model and values are examined
for general consistency with biological aspects of pollock (which are
known to vary in proximity to the bottom with age and between years).

\#\#\#Comments specific to this assessment In the September 2016
minutes, the BSAI Plan Team recommended: ``\ldots{} that the authors
develop a better prior for steepness, or at least a better rationale,
and perhaps consider a meta-analytic approach. The Team recommends using
biomass in the AT and BTS (his Model 4 in the presentation), which also
includes the bottom 2.5 m of the acoustic biomass. In the long term, the
Team recommends evaluating the sample sizes used for the data weighting
and pursuing other CIE suggestions.

The AT and BTS data are treated as biomass indices in this assessment.
Sample size estimates were re-evaluated and used in the recommended
model below. An alternative degree of uncertainty, which notes
differences from the CEATTLE stock-recruit relationship was provided as
an alternative (but is unfortunately lacking in meta-analytic rigor).
The age compositions for including the bottom 2.5 meters from the
acoustic data were unavailable in time for this assessment and will be
applied in the coming year.

\#Introduction Walleye pollock (Gadus chalcogrammus; hereafter referred
to as pollock) are broadly distributed throughout the North Pacific with
the largest concentrations found in the Eastern Bering Sea. Also
marketed under the name Alaska pollock, this species continues to
represent over 40\% of the global whitefish production, with the market
disposition split fairly evenly between fillets, whole fish (headed and
gutted), and surimi (Fissel et al.~2014). An important component of the
commercial production is the sale of roe from pre-spawning pollock.
Pollock are considered a relatively fast growing and short-lived
species. They play an important role in the Bering Sea ecosystem.

\#\#Stock structure A summary of EBS pollock stock structure was
presented at the September 2015 BSAI Plan Team meetings. From that
review the Team and SSC concurred that the current stock structure
hypothesis for management purposes was of little or no concern.

\#Fishery

EBS pollock catches were low until directed foreign fisheries began in
1964. Catches increased rapidly during the late 1960s and reached a peak
in 1970-75 when they ranged from 1.3 to 1.9 million t annually (Fig.
1.1). Following the peak catch in 1972, bilateral agreements with Japan
and the USSR resulted in reductions. Since 1977 (when the U.S. EEZ was
declared) the annual average EBS pollock catch has been about 1.2
million t, ranging from 0.815 million t in 2009 to nearly 1.5 million t
during 2003-2006 (Fig. 1.1). United States vessels began fishing for
pollock in 1980 and by 1987 they were able to take 99\% of the quota.
Since 1988, only U.S. vessels have been operating in this fishery.
Observers collected data aboard the foreign vessels since the late
1970s. The current observer program for the domestic fishery formally
began in 1991 and has since then regularly re-evaluated the sampling
protocol and making adjustments where needed to improve efficiency.
Since 2011, regulations require that all vessels participating in the
pollock fishery carry at least one observer. Prior to this time about
70-80\% of the catch was observed at sea or during dockside offloading.
During a 10-year period, catches by foreign vessels operating in the
``Donut Hole'' region of the Aleutian Basin were substantial totaling
nearly 7 million t (Table 1.1). A fishing moratorium was enacted in 1993
and only trace amounts of pollock have been harvested from the Aleutian
Basin region since then.

\#\#Management measures

The EBS pollock stock is managed by NMFS regulations that provide limits
on seasonal catch. The NMFS observer program data provide near real-time
statistics during the season and vessels operate within well-defined
limits. TACs have commonly been set well below the ABC value and catches
have usually stayed within these constraints (Table 1.2). Allocations of
the TAC split first with 10\% to western Alaska communities as part of
the Community Development Quota (CDQ) program and the remainder between
at-sea processors and shore-based sectors. In recent studies, Haynie
(2014) characterized the CDQ program and Seung and Ianelli (2016)
combine a fish population dynamics model with an economic model to
evaluate regional impacts.

Due to concerns over possible impacts groundfish fisheries may have on
rebuilding populations of Steller sea lions, a number of management
measures have been implemented. Some measures were designed to reduce
the possibility of competitive interactions between fisheries and
Steller sea lions. For the pollock fisheries, seasonal fishery catch and
pollock biomass distributions (from surveys) indicated that the apparent
disproportionately high seasonal harvest rates within Steller sea lion
critical habitat could lead to reduced sea lion prey densities.
Consequently, management measures redistributed the fishery both
temporally and spatially according to pollock biomass distributions.
This was intended to disperse fishing so that localized harvest rates
were more consistent with annual exploitation rates. The measures
include establishing: 1) pollock fishery exclusion zones around sea lion
rookery or haulout sites; 2) phased-in reductions in the seasonal
proportions of TAC that can be taken from critical habitat; and 3)
additional seasonal TAC releases to disperse the fishery in time.

Prior to adoption of the above management measures, the pollock fishery
occurred in each of the three major NMFS management regions of the North
Pacific Ocean: the Aleutian Islands (1,001,780 km2 inside the EEZ), the
Eastern Bering Sea (968,600 km2), and the Gulf of Alaska (1,156,100
km2). The marine portion of Steller sea lion critical habitat in Alaska
west of 150°W encompasses 386,770 km2 of ocean surface, or 12\% of the
fishery management regions.

Prior to 1999, 84,100 km2, or 22\% of critical habitat was closed to the
pollock fishery. Most of this closure consisted of the 10- and 20-nm
radius all-trawl fishery exclusion zones around sea lion rookeries
(48,920 km2, or 13\% of critical habitat). The remainder was largely
management area 518 (35,180 km2, or 9\% of critical habitat) that was
closed pursuant to an international agreement to protect spawning stocks
of central Bering Sea pollock.

In 1999, an additional 83,080 km2 (21\%) of critical habitat in the
Aleutian Islands was closed to pollock fishing along with 43,170 km2
(11\%) around sea lion haulouts in the GOA and Eastern Bering Sea. In
1998, over 22,000 t of pollock were caught in the Aleutian Island
region, with over 17,000 t taken within critical habitat region. Between
1999 and 2004 a directed fishery for pollock was prohibited in this
region. Subsequently, 210,350 km2 (54\%) of critical habitat was closed
to the pollock fishery. In 2000 the remaining phased-in reductions in
the proportions of seasonal TAC that could be caught within the BSAI
Steller sea lion Conservation Area (SCA) were implemented.

On the EBS shelf, an estimate (based on observer at-sea data) of the
proportion of pollock caught in the SCA has averaged about 38\%
annually. During the A-season, the average is about 42\% (in part
because pre-spawning pollock are more concentrated in this area during
this period). The proportion of pollock caught within the SCA varies
considerably, presumably due to temperature regimes and population age
structure. The annual proportion of catch within the SCA varies and has
ranged from an annual low of 11\% in 2010 to high of 51\% in 2016 (Table
1.3). This high value was due to B-season conditions which had 62\% of
the catch taken in this region.

The 1998 American Fisheries Act (AFA) reduced the capacity of the
catcher/processor fleet and permitted the formation of cooperatives in
each industry sector by the year 2000. Because of some of its
provisions, the AFA gave the industry the ability to respond efficiently
to changes mandated for sea lion conservation and salmon bycatch
measures. Without such a catch-share program, these additional measures
would likely have been less effective and less economical (Strong and
Criddle 2014).

An additional strategy to minimize potential adverse effects on sea lion
populations is to disperse the fishery throughout more of the pollock
range on the Eastern Bering Sea shelf. While the distribution of fishing
during the A-season is limited due to ice and weather conditions, there
appears to be some dispersion to the northwest area (Fig. 1.3).

The majority (\textasciitilde{}56\%) of Chinook salmon caught as bycatch
in the pollock fishery originate from western Alaskan rivers. An
Environmental Impact Statement (EIS) was completed in 2009 in
conjunction with the Council's recommended management approach. This EIS
evaluated the relative impacts of different bycatch management
approaches as well as estimated the impact of bycatch levels on adult
equivalent salmon (AEQ) returning to river systems (NMFS/NPFMC 2009). As
a result, revised salmon bycatch management measures went into effect in
2011imposing prohibited species catch (PSC) limits that when reached
would close the fishery by sector and season (Amendment 91 to the
Groundfish FMP resulting from the NPFMC's 2009 action). Previously, all
measures for salmon bycatch imposed seasonal area closures when PSC
levels reached the limit (fishing could continue outside of the closed
areas). The new program imposes a dual cap system broken out by fishing
sector and season. The management measure was designed to keep the
annual bycatch below the lower cap by providing incentives to avoid
bycatch. Additionally, in order to participate, vessels must take part
in an incentive program agreement (IPA). These IPAs are approved by NMFS
and are designed for further bycatch reduction and individual vessel
accountability. The fishery has been operating under rules to implement
this program since January 2011. During 2008 - 2016, bycatch levels for
Chinook salmon have been well below average following record high levels
in 2007. This is likely due to industry-based restrictions on areas
where pollock fishing may occur, environmental conditions, Amendment 91
measures, and salmon abundance.

Further measures to reduce salmon bycatch in the pollock fishery were
developed and the Council took action on Amendment 110 to the BSAI
Groundfish FMP in April 2015. These additional measures were designed to
add protection for Chinook salmon by imposing more restrictive PSC
limits in times of low western Alaskan Chinook salmon abundance. This
included provisions within the IPAs that reduce fishing in months of
higher bycatch encounters and mandate the use of salmon excluders in
trawl nets. These provisions were also included to manage chum salmon
bycatch within the IPAs rather than through Amendment 84 to the FMP. The
new measure also included additional seasonal flexibility in pollock
fishing so that more pollock (proportionally) could be caught during
seasons when salmon bycatch rates were low. Specifically, an additional
5\% of the pollock can be caught in the Aseason (effectively changing
the seasonal allocation from 40\% to 45\%). These measures are all part
of Amendment 110 and a summary of this and other key management measures
is provided in Table 1.4.

\#Fishery characteristics \#\#General catch patterns

The ``A-season'' for directed EBS pollock fishing opens on January 20th
and extends into early-mid April. During this season, the fishery
produces highly valued roe that, under optimal conditions, can comprise
over 4\% of the catch in weight. The second, or ``B-season'' presently
opens on June 10th and extends through noon on November 1st. The
A-season fishery concentrates primarily north and west of Unimak Island
depending on ice conditions and fish distribution. There has also been
effort along the 100 m contour (and deeper) between Unimak Island and
the Pribilof Islands. Since 2011, regulations and industry-based
measures to reduce salmon bycatch have affected the spatial distribution
of the fishery and to some degree, the way individual vessel operators
fish (Stram and Ianelli, 2014). The catch estimates by sex for the
A-season compared to estimates for the entire season indicate that over
time, the number of males and females has been fairly equal (Fig. 1.2).
The 2016 and 2014 A-season fishery spatial pattern had relatively high
concentrations of fishing on the shelf north of Unimak Island,
especially compared to the pattern observed in 2015 when most fishing
activity occurred farther north (Fig. 1.3).

The 2016 summer and fall (B-season) fishing continued the trend of
fleet-wide higher catch per hour fished (Fig. 1.4). Compared to 2011
B-season, the combined fleet took about one third of the actual fishing
time to reach 600 kt. Spatially, the 2016 B-season was much more
concentrated around the ``horseshoe,'' near the shelf break west of the
Pribilof Islands and extending north and west from Amak Island (Fig.
1.5). Since 1979 the catch of EBS pollock has averaged 1.19 million t
with the lowest catches occurring in 2009 and 2010 when the limits were
set to 0.81 million t due to stock declines (Table 1.1

Pollock retained and discarded catch (based on NMFS observer estimates)
in the Eastern Bering Sea and Aleutian Islands for 1991-2016 are shown
in Table 1.5. Since 1991, estimates of discarded pollock have ranged
from a high of 9.1\% of total pollock catch in 1992 to recent lows of
around 0.6\%. These low values reflect the implementation of the
Council's Improved Retention /Improved Utilization program. Prior to the
implementation of the AFA in 1999, higher discards may have occurred
under the ``race for fish'' and incidental catch of pollock that were
below marketable sizes. Since implementation of the AFA, the vessel
operators have more time to pursue optimal sizes of pollock for market
since the quota is allocated to vessels (via cooperative arrangements).
In addition, several vessels have made gear modifications to avoid
retention of smaller pollock. In all cases, the magnitude of discards
counts as part of the total catch for management (to ensure the TAC is
not exceeded) and within the assessment. Bycatch of other non-target,
target, and prohibited species is presented in the section titled
Ecosystem Considerations below. In that section it is noted that the
bycatch of pollock in other target fisheries is more than double the
bycatch of other target species (e.g., Pacific cod) in the pollock
fishery.

\#\#Economic conditions as of 2015

Alaska pollock is the dominant species in terms of catch in the Bering
Sea and Aleutian Island (BSAI) region. It accounted for 69\% of the
BSAI's FMP groundfish harvest and 89\% of the total pollock harvest in
Alaska. Retained catch of pollock increased 2.2\% to 1.3 million t in
2015. BSAI pollock first- wholesale value was \$1.28 billion 2015, which
was down slightly from \$1.3 billion in 2014 but above the 2005-2007
average of \$1.25 billion. The higher revenue in recent years is largely
the result of increased catch and production levels as the average
first-wholesale price of pollock products have declined since peaking in
2008-2010 and since 2013 have been below the 2005-2007 average, though
this varies across products types.

Pollock is targeted exclusively with pelagic trawl gear. The catch of
pollock in the BSAI was rationalized with the passage of the American
Fisheries Act (AFA) in 1998, which, among other things, established a
proportional allocation of the total allowable catch (TAC) among vessels
in sectors which could form into cooperatives. Alaska-caught pollock in
the BSAI became certified by the Marine Stewardship Council (MSC) in
2005, a NGO based third-party sustainability certification, which some
buyers seek. In 2015 the official U.S. market name changed from ``Alaska
pollock'' to ``pollock'' enabling U.S. retailers to differentiate
between pollock caught in Alaska and Russia.

Prior to 2008 pollock catches were high at approximately 1.4 million t
in the BSAI for an extended period (Tables 1.6). The U.S. accounted for
over 50\% of the global pollock catch (Table 1.7). Between 2008-2010
conservation reductions in the pollock total allowable catch (TAC)
trimmed catches to an average 867 thousand t. The supply reduction
resulted in price increases for most pollock products, which mitigated
the short-term revenue loss (Table 1.8). Over this same period, the
pollock catch in Russia increased from an average of 1 million t in
2005-2007 to 1.4 million t in 2008-2010 and Russia's share of global
catch increased to over 50\% and the U.S. share decreased to 35\%.
Russia lacks the primary processing capacity of the U.S. and much of
their catch is exported to China and is re-processed as twice-frozen
fillets. Around the mid- to late-2000s, buyers in Europe, an important
segment of the fillet market, started to source fish products with the
MSC sustainability certification, and some major retailer in the U.S.
later began to follow suit. Asian markets, an important export
destination for several pollock products, have shown less interest in
requiring MSC certification. The U.S. was the only producer of MSC
certified pollock until 2013 when roughly 50\% of the Russian catch
became MSC certified. Since 2010 the U.S. pollock stock rebounded with
catches in the BSAI ranging from 1.2-1.3 million t and Russia's catch
has stabilized at 1.5 to 1.6 million t. Most pollock are exported;
consequently, exchange rates can have a significant impact on market
dynamics, particularly the Dollar-Yen and Dollar-Euro. Additionally,
pollock more broadly competes with other whitefish that, to varying
degrees, can serve as substitutes depending on the product.

This market environment accounts for some of the major trends in prices
and production across product types. Fillet prices peaked in 2008-2010
but declined afterwards because of the greater supply from U.S. and
Russia. The 2013 MSC certification of Russian-caught pollock enabled
access to segments of European and U.S. fillet markets, which has put
continued downward pressure on prices. Pollock roe prices and production
have declined steadily over the last decade as international demand has
waned with changing consumer preferences in Asia. Additionally, the
supply of pollock roe from Russia has increased with catch. The net
effect has been not only a reduction in the supply of roe from the U.S.
industry, but also a significant reduction in roe prices which are
roughly half pre-2008 levels. Prior to 2008, roe comprised 23\% of the
U.S. wholesale value share, and since 2011 it has been roughly 10\%.
Within the U.S. the supply reduction in 2008-2010 surimi production from
pollock came under increased pressure as U.S. pollock prices rose and
markets sought cheaper sources of raw materials. This contributed to a
growth in surimi from warm-water fish of southeast Asia. Surimi prices
spiked in 2008-2010 and have since tapered off as production from
warm-water species increased (as has pollock). A relatively small
fraction of pollock caught in Russian waters is processed as surimi.
Surimi is consumed globally, but Asian markets dominate the demand for
surimi and demand has remained strong.

The catch of pollock can be broadly divided between the shore-based
sector where catcher vessels make deliveries to inshore processors, and
the at-sea sector where catch is processed at-sea by catcher/processors
and motherships before going directly to the wholesale markets. The
retained catch of the shore-based sector increased 3\% increase to 687
thousand t. The value of these deliveries (shore-based ex-vessel value)
totaled \$227.3 million in 2015, which was roughly equal to the
shore-based ex-vessel value in 2014, as the increased catch was offset
by similar decrease in the ex-vessel price. The first-wholesale value of
pollock products was \$768 million for the at-sea sector and \$516
million for the shore-based sector. The higher revenue in recent years
is largely the result of increased catch levels as the average price of
pollock products have declined since peaking in 2008-2010 and since 2013
have been below the 2005-2007 average, though this varies across
products types. The average price of pollock products in 2015 increased
slightly for the at-sea sector and decreased slightly for the
shore-based sector, which was attributable to sectoral differences in
price change of fillet and surimi products.

The portfolios of products shore-based and at-sea processors produce are
similar. In both sectors the primary products processed from pollock are
fillets, surimi and roe, with each accounting for approximately 40\%,
35\%, and 10\% of first-wholesale value. The price of products produced
at-sea tend to be higher than comparable products produced shore-based
because of the shorter time span between catch, processing and freezing.
The price of fillets produced at-sea tend to be about 10\% higher,
surimi prices tend to be about 20\% higher and the price of roe about
40\% higher. Average prices for fillets produced at-sea also tend to be
higher because they produce proportionally more higher-priced fillet
types (like deep-skin fillets). The at-sea price first wholesale premium
averaged roughly \$0.30 per pound between 2005-2010 but has decreased to
an average of \$0.19 per pound since 2011, in part, because the
shore-based sector increased their relative share of surimi production.

A variety of different fillets are produced from pollock, with
pin-bone-out (PBO) and deep-skin fillets accounting for approximately
70\% and 30\% of production in the BSAI, respectively. Total fillet
production decreased 5\% to 167 thousand t in 2015, but since 2010 has
increased with aggregate production and catch and has been higher than
the 2005-2007 average. The average price of fillet products in the BSAI
decreased 1\% to \$1.35 per pound and is below the inflation adjusted
average price of fillets in 2005-2007 of \$1.44 per pound. Price
negotiations with European buyers in 2015 were difficult with buyers
citing exchange rates as an impediment. While still a small portion of
their primary production, Russia producers increased fillet production
in 2015 and report plans to upgrade their production capacity in the
near future. Much of the Russian catch already goes to China for
secondary processing into fillets so this would do little to increase
the overall volume, however, increased primary fillet processing in
Russia could increase competition with U.S. produced single-frozen
fillet products. Approximately 30\% of the fillets produced in Alaska
are estimated to remain in the domestic market, which accounts for
roughly 45\% of domestic pollock fillet consumption. As recent fillet
markets have become increasingly tight, the industry has tried to
maintain value by increasing domestic marketing for fillet based product
and creating product types that are better suited to the American
palette, in addition to increased utilization of by-products.

\#\#\#Surimi seafood Surimi production continued an increasing trend
through 2015, rising 10\% to 187.7 thousand t which is above the
2005-2007 average. Prices have increased since 2013 to an average of
\$1.14 per pound in the BSAI in 2015. The production and price increase
in 2015 were attributable to a reduction in the international supply of
surimi, particularly from Thailand, that reduced Japanese inventories.
Because surimi and fillets are both made from pollock meat, activity in
the fillet market can influence the decision of processors to produce
surimi. The difficulties in the European fillet market in 2015 further
incentivized the shift in production from fillets to surimi.
Additionally, industry news indicated a decrease in the average size of
fish caught, which yield higher value when processed as surimi than
fillets.

\#\#\#Pollock roe Roe is a high priced product that is the focus of the
A-season catch destined primarily for Asian markets. Roe production in
the BSAI tapered off in the late-2000s and since has generally
fluctuated at under 20 thousand t annually, production averaged 27
thousand t in 2005-2007 and was 19 thousand t in 2015 (Fig. 1.6). Prices
peaked in the mid-2000s prices and have decreased over the last decade
through 2015 (prices dropped 21\% to \$2.30 per pound). The weakness in
the Yen against the U.S. Dollar has been cited as a factor in the 2015
price drop. Additionally, the Japanese Yen has remained strong against
the Russian Ruble, which makes Russian products relatively cheaper than
U.S. products for Japanese buyers. Also, the production volume from
Russia has contributed to a carryover of roe inventory in Asian markets,
which puts downward pressure on prices. Industry reports further
indicate that harvests yielded comparatively more over-mature lower
grade roe in 2015 which also contributed to low prices. In terms of
recent trends, overall roe production declined with the catch limits
during 2007-2010 while the B-season production remained relatively flat
until 2015 and 2016 (Fig. 1.6). This is likely due to the fish size and
perhaps warmer conditions.

\#\#\#Fish oil Using oil production per ton as a basic index (tons of
oil per ton of retained catch) shows increases for the at-sea sector. In
2005-2007 it was 0.3\% and starting in 2008 it increased and leveled off
around 2010 with a little over 1.5\% of the catch being converted to
fish oil (Table 1.9). This represents about a 5-fold increase in
recorded oil production during this period. Oil production from the
shore-based fleet was somewhat higher than the at-sea processors prior
to 2008 but has been relatively stable according to available records.
Oil production estimates from the shore-based fleet may be biased low
because some production occurs at secondary processors (fishmeal plants)
in Alaska. The increased production of oil beginning in 2008 can be
attributed to the steady trend to add more value per ton of fish landed.

\#Data The following lists the data used in the assessment:

\begin{longtable}[]{@{}lll@{}}
\toprule
\begin{minipage}[b]{0.26\columnwidth}\raggedright
Source\strut
\end{minipage} & \begin{minipage}[b]{0.22\columnwidth}\raggedright
Type\strut
\end{minipage} & \begin{minipage}[b]{0.43\columnwidth}\raggedright
Years\strut
\end{minipage}\tabularnewline
\midrule
\endhead
\begin{minipage}[t]{0.26\columnwidth}\raggedright
Fishery\strut
\end{minipage} & \begin{minipage}[t]{0.22\columnwidth}\raggedright
Catch biomass\strut
\end{minipage} & \begin{minipage}[t]{0.43\columnwidth}\raggedright
1964-2017\strut
\end{minipage}\tabularnewline
\begin{minipage}[t]{0.26\columnwidth}\raggedright
Fishery\strut
\end{minipage} & \begin{minipage}[t]{0.22\columnwidth}\raggedright
Catch age composition\strut
\end{minipage} & \begin{minipage}[t]{0.43\columnwidth}\raggedright
1964-2016\strut
\end{minipage}\tabularnewline
\begin{minipage}[t]{0.26\columnwidth}\raggedright
Fishery\strut
\end{minipage} & \begin{minipage}[t]{0.22\columnwidth}\raggedright
Japanese trawl CPUE\strut
\end{minipage} & \begin{minipage}[t]{0.43\columnwidth}\raggedright
1965-1976\strut
\end{minipage}\tabularnewline
\begin{minipage}[t]{0.26\columnwidth}\raggedright
EBS bottom trawl\strut
\end{minipage} & \begin{minipage}[t]{0.22\columnwidth}\raggedright
Area-swept biomass and age-specific proportions\strut
\end{minipage} & \begin{minipage}[t]{0.43\columnwidth}\raggedright
1982-2017\strut
\end{minipage}\tabularnewline
\begin{minipage}[t]{0.26\columnwidth}\raggedright
Acoustic trawl survey\strut
\end{minipage} & \begin{minipage}[t]{0.22\columnwidth}\raggedright
Biomass index and age-specific proportions\strut
\end{minipage} & \begin{minipage}[t]{0.43\columnwidth}\raggedright
1994, 1996, 1997, 1999, 2000, 2002, 2004, 2006-2010, 2012, 2014,
2016\strut
\end{minipage}\tabularnewline
\begin{minipage}[t]{0.26\columnwidth}\raggedright
Acoustic vessels of opportunity (AVO)\strut
\end{minipage} & \begin{minipage}[t]{0.22\columnwidth}\raggedright
Biomass index\strut
\end{minipage} & \begin{minipage}[t]{0.43\columnwidth}\raggedright
2006-2017\strut
\end{minipage}\tabularnewline
\bottomrule
\end{longtable}

\#\#Fishery

The catch-at-age composition was estimated using the methods described
by Kimura (1989) and modified by Dorn (1992). Length-stratified age data
are used to construct age-length keys for each stratum and sex. These
keys are then applied to randomly sampled catch length frequency data.
The stratum-specific age composition estimates are then weighted by the
catch within each stratum to arrive at an overall age composition for
each year. Data were collected through shore-side sampling and at-sea
observers. The three strata for the EBS were: i) January--June (all
areas, but mainly east of 170°W); ii) INPFC area 51 (east of 170°W) from
July--December; and iii) INPFC area 52 (west of 170°W) from
July--December. This method was used to derive the age compositions from
1991-2015 (the period for which all the necessary information is readily
available). Prior to 1991, we used the same catch-at-age composition
estimates as presented in Wespestad et al.~(1996).

The catch-at-age estimation method uses a two-stage bootstrap
re-sampling of the data. Observed tows were first selected with
replacement, followed by re-sampling actual lengths and age specimens
given that set of tows. This method allows an objective way to specify
the effective sample size for fitting fishery age composition data
within the assessment model. In addition, estimates of stratum-specific
fishery mean weights-at-age (and variances) are provided which are
useful for evaluating general patterns in growth and growth variability.
For example, Ianelli et al.~(2007) showed that seasonal aspects of
pollock condition factor could affect estimates of mean weight-at-age.
They showed that within a year, the condition factor for pollock varies
by more than 15\%, with the heaviest pollock caught late in the year
from October-December (although most fishing occurs during other times
of the year) and the thinnest fish at length tending to occur in late
winter. They also showed that spatial patterns in the fishery affect
mean weights, particularly when the fishery is shifted more towards the
northwest where pollock tend to be smaller at age. In 2011 the winter
fishery catch consisted primarily of age 5 pollock (the 2006 year class)
and later in that year age 3 pollock (the 2008 year class) were present.
In 2012 - 2015 the 2008 year class been prominent in the catches with
2015 showing the first signs of the 2012 year-class as three year-olds
in the catch (Fig. 1.7; Table 1.10). The sampling effort for age
determinations and lengths is shown in Tables 1.11 and 1.12. Sampling
for pollock lengths and ages by area has been shown to be relatively
proportional to catches (e.g., Fig. 1.8 in Ianelli et al.~2004). As part
of the re-evaluation of sample sizes assumed within the assessment, the
number of ages and lengths (and number of hauls from which samples were
collected) show significant changes over time (Fig. 1.8). This
information was used to inform periods from which input sample size
re-weighting was appropriate for modeling. Regarding the precision of
total pollock catch biomass, Miller (2005) estimated the CV to be on the
order of 1\%.

Scientific research catches are reported to fulfill requirements of the
Magnuson-Stevens Fisheries Conservation and Management Act. The annual
estimated research catches (1963 - 2015) from NMFS surveys in the Bering
Sea and Aleutian Islands Region are given in Table 1.13. Since these
values represent extremely small fractions of the total removals
(\textasciitilde{}0.02\%) they are ignored as a contributor to the
catches as modeled for assessment purposes.

\#Surveys \#\#Bottom trawl survey (BTS) Trawl surveys have been
conducted annually by the AFSC to assess the abundance of crab and
groundfish in the Eastern Bering Sea since 1979 and since 1982 using
standardized gear and methods. For pollock, this survey has been
instrumental in providing an abundance index and information on the
population age structure. This survey is complemented by the acoustic
trawl (AT) surveys that sample mid-water components of the pollock
stock. Between 1991 and 2016 the BTS biomass estimates ranged from 2.28
to 8.39 million t (Table 1.14; Fig. 1.9). In the mid-1980s and early
1990s several years resulted in above-average biomass estimates. The
stock appeared to be at lower levels during 1996-1999 then increased
moderately until about 2003 and since then has averaged just over 4
million t. These surveys provide consistent measurements of
environmental conditions, such as the sea surface and bottom
temperatures. Large-scale zoogeographic shifts in the EBS shelf
documented during a warming trend in the early 2000s were attributed to
temperature changes (e.g., Mueter and Litzow 2008). However, after the
period of relatively warm conditions ended in 2005, the next eight years
were mainly below average, indicating that the zoogeographic responses
may be less temperature-dependent than they initially appeared (Kotwicki
and Lauth 2013). Bottom temperatures increased in 2011 to about average
from the low value in 2010 but declined again in 2012-2013. However, in
2014-2015 bottom temperatures have increased along with surface
temperatures and have reached a new high in 2016 (Fig. 1.10).

Beginning in 1987 NMFS expanded the standard survey area farther to the
northwest. The pollock biomass levels found in the two northern strata
were highly variable, ranging from 1\% to 22\% of the total biomass;
whereas the 2014 estimate was 12\%, 2015 was 7\%, and this year (2016)
slightly below the average (5\%) at 4\% (Table 1.15). In some years
(e.g., 1997 and 1998) some stations had high catches of pollock in that
region and this resulted in high estimates of sampling uncertainty (CVs
of 95\% and 65\% for 1997 and 1998 respectively). This region is
contiguous with the Russian border and these strata seem to improve
coverage over the range of the exploited pollock stock.

The 2016 biomass estimate (design-based, area swept) was 4.91 million t,
slightly above the average for this survey (4.84~million t). Pollock
were distributed more patchily in 2016 than in recent years and were
most concentrated in the outer domain, relatively unconstrained by the
warmer bottom temperatures (Fig. 1.11). The spatial distribution of
pollock densities in the 2016 survey appeared to be split with high
densities in the southeast and northwest of the main survey area with a
gap about one third of the distance from north to south (Fig. 1.12).

The BTS abundance-at-age estimates shows variability in year-class
strengths with substantial consistency over time (Fig. 1.13). Pollock
above 40 cm in length generally appear to be fully selected and in some
years many 1-year olds occur on or near the bottom (with modal lengths
around 10-19 cm). Age 2 or 3 pollock (lengths around 20-29 cm and 30-39
cm, respectively) are relatively rare in this survey presumably due to
off-bottom distributions. Observed fluctuations in survey estimates may
be attributed to a variety of sources including unaccounted-for
variability in natural mortality, survey catchability, and migrations.
As an example, some strong year classes appear in the surveys over
several ages (e.g., the 1989 year class) while others appear only at
older ages (e.g., the 1992 and 2008 year class). Sometimes initially
strong year classes appear to wane in successive assessments (e.g., the
1996 year class estimate (at age 1) dropped from 43 billion fish in 2003
to 32 billion in 2007 (Ianelli et al.~2007). Retrospective analyses
(e.g., Parma 1993) have also highlighted these patterns, as presented in
Ianelli et al.~(2006, 2011). Kotwicki et al.~(2013) also found that that
the catchability of either BTS or AT survey for pollock is variable in
space and time because it depends on environmental variables, and is
density-dependent in the case of the BTS survey.

The 2016 survey age compositions were developed from age-structures
collected during the survey (June-July) and processed at the AFSC labs
within a few weeks after the survey was completed. The level of sampling
for lengths and ages in the BTS is shown in Table 1.16. The estimated
numbers-at-age from the BTS for strata (1-9 except for 1982-84 and 1986,
when only strata 1-6 were surveyed) are presented in Table 1.17. Table
1.18 contains the values used for the index which accounts for
density-dependence in bottom trawl tows (Kotwicki et al.~2014). Mean
body mass at ages from the survey are shown in Table 1.19.

As in previous assessments, a descriptive evaluation of the BTS data
alone was conducted to examine mortality patterns similar to those
proposed in Cotter et al.~(2004). The idea is to evaluate survey data
independently from the assessment model for trends. The log-abundance of
age 5 and older pollock was regressed against age by cohort. The
negative values estimated for the slope are estimates of total annual
mortality. Age-5 was selected because younger pollock appear to still be
recruiting to the bottom trawl survey gear (based on qualitative
evaluation of age composition patterns). A key assumption of this
analysis is that all ages are equally available to the gear. Total
mortality by cohort seems to be variable (unlike the example in Cotter
et al., 2004). Cohorts from the early 1990s appear to have lower total
mortality than cohorts since the mid-1990s, which average around 0.4
(Fig. 1.14). Total mortality estimates by cohort represent lifetime
averages since harvest rates (and actual natural mortality) vary from
year to year. The low values estimated for some year classes (e.g., the
1991 cohort) could be because these age groups only become available to
the survey at a later age (i.e., that the availability/selectivity to
the survey gear changed for these cohorts). Alternatively, it may
suggest some net immigration into the survey area or a period of lower
natural mortality. In general, these values are consistent with the
values obtained within the assessment models.

As described in the 2015 assessment, an alternative index that accounts
for the efficiency of bottom-trawl gear for estimating pollock densities
was used (Kotwicki et al.~2014). Based on comments from the CIE review,
this index was provided in biomass units in this assessment (previously
the index was for abundance).

\#Other time series used in the assessment

\#\#Acoustic trawl (AT) surveys

The AT surveys are conducted biennially and are designed to estimate the
off- bottom component of the pollock stock (compared to the BTS which
are conducted annually and provide an abundance index of the near-bottom
pollock). The number of trawl hauls, lengths, and ages sampled from the
AT survey are presented in Table 1.20. Estimated midwater pollock
biomass for the shelf was above 4 million tons in the early years of the
time series (Table 1.14). It dipped below 2 million t in 1991, and then
increased and remained between 2.5 and 4 million t for about a decade
(1994-2004). The early 2000s (the `warm' period mentioned above) were
characterized by low pollock recruitment, which was subsequently
reflected in lower midwater biomass estimates between 2006 and 2012 (the
recent `cold' period; Honkalehto and McCarthy 2015). The midwater
pollock biomass estimate from the 2016 AT survey of 4.06 million is
above the average (2.76 million t). Previously relative estimation
errors for the total biomass were derived from a one-dimensional (1D)
geostatistical method (Petitgas 1993, Walline 2007, Williamson and
Traynor 1996). This method accounts for observed spatial structure for
sampling along transects. As in previous assessments, the other sources
of error (e.g., target strength, trawl sampling) were accounted for by
inflating the annual error estimates to have an overall average CV of
25\% for application within the assessment model (based on judgement
relative to other indices).

The 2016 summer AT survey age compositions were developed using an
age-length key from the BTS supplemented with a sample of 100 AT survey
juveniles (\textless{}38 cm fork length) to fill in size classes not
well sampled by the BTS (Fig. 1.15; Table 1.21). Of particular note was
very few age 1 pollock were found whereas age 3 (the 2013 year class)
was the most abundant age group followed by four year olds. Spatially,
the 2016 mid-water pollock distribution was somewhat consistent with
recent years. The portion of shelf-wide biomass estimated to be east of
170º W was 37\%, compared to an average of 24\% since 1994 (Table 1.22).
Also, the distribution of pollock biomass within the SCA was similar to
that found in 2014 at 13\% compared to the 2007-2012 average of 7\% (and
1994-2016 average of 10\%).

\#\#Biomass index from Acoustic-Vessels-of-Opportunity (AVO)

The details of how acoustic backscatter data from the two commercial
fishing vessels chartered for the eastern Bering Sea bottom trawl (BT)
survey are used to compute a midwater abundance index for pollock can be
found in Honkalehto et al.~2011. This index is updated during years when
a directed acoustic-trawl survey is not carried out in the EBS to
provide an additional source of information on pollock found in
mid-water. The most recent update was in 2015 when opportunistic data in
2014 and 2015 were compiled and used within the assessment (due to
research staff issues when a full AT survey is conducted, the AVO data
are processed in years when the RV Oscar Dyson is working in other
regions, i.e., in ``off years'' for the AT survey). The series used for
this assessment shows a steady increase for the period 2009-2015 (Table
1.23; Honkalehto et al.~in review).

A spatial comparison between the BTS data and AT survey transects in
2014 and 2016 shows differences in the locales and densities of pollock
both between years and in their vertical densities within years (Fig.
1.16). This figure also shows that in 2016, the AT survey densities were
higher over a larger area than in 2014 while for the BTS data, there
appears to be more of a distinct separation between the southeast
aggregation and the northeast portion of the shelf. Also, an unusual
occurrence of good pollock densities was found in the inner domain into
Bristol Bay and nearer Nunivak Island than usual.

\#Analytic approach \#\#Model structure A statistical age-structured
assessment model conceptually outlined in Fournier and Archibald (1982)
and like Methot's (1990) stock synthesis model was applied over the
period 1964-2016. A technical description is presented in the Model
Details section. The analysis was first introduced in the 1996 SAFE
report and compared to the cohort analyses that had been used previously
and was document Ianelli and Fournier 1998). The model was implemented
using automatic differentiation software developed as a set of libraries
under the C++ language (``ADMB,'' Fournier et al.~2012). The data
updated from last year's analyses include:

\begin{itemize}
\tightlist
\item
  The `r thisyr' EBS bottom trawl survey estimates of population
  numbers-at-age was added and biomass.\\
\item
  The `r thisyr' EBS acoustic-trawl survey estimate of population
  numbers-at-age based on the age data from the BTS survey for the
  age-length key for the AT survey.\\
\item
  The `r thisyr-1' fishery age composition data were added.
\end{itemize}

A simplified version of the assessment (with mainly the same data and
likelihood-fitting method) is included as a supplemental multi-species
assessment model. Importantly, it allows for trophic interactions with
key predators for pollock and can be used to evaluate age and
time-varying natural mortality estimates in addition to alternative
catch scenarios and management targets (see this volume:
\url{http://www.afsc.noaa.gov/refm/stocks/plan_team/EBSmultispp.pdf}).

\#\#Description of alternative models

Based on the CIE review, a few model configuration options were
developed and implemented. To match these features with model names the
following table is for descriptive purposes. Note that Models 16.0x were
considered preliminary for investigation and sensitivity to changes. At
the September 2016 Plan Team meetings and subsequent SSC presentations
were made describing preliminary results using the ATS data that covered
the water column down to 0.5m from the bottom. Due to issues with
compiling the age compositions for the new series, the plan is to
incorporate and present these results in the 2017 assessment.

\#\#\#Input sample size As part of the CIE review recommendation, the
assessment was reevaluated against specified sample sizes and
flexibility of time and age varying selectivity. The first phase
proceeded as in the past to specify that the fishery average input
sample size was equivalent to about 350 fish for the recent era (since
1991) and lower values for the intermediate and earliest period (as
shown in Table 1.24). We assumed average values of 100 and 50 for the
BTS and ATS data, respectively and modified so that the inter-annual
variability reflected the variability in the number of hauls sampled.
For model 16.03, effective sample size weights were estimated following
Francis 2011 (equation TA1.8, hereafter referred to as Francis weights)
computed for the BTS and ATS composition data and over three stanzas of
fishery data: from 1964-1976, 1977-1998, and 1999-2015. The
justification for breaking the fishery estimates into these periods
reflects the different data sources and/or sampling programs from which
catch-age information was compiled. Under these assumptions, we modified
the sample sizes for the recent two periods according to the estimated
Francis weights. The estimated multipliers for the early period
suggested increasing the sample size. However, since these data occur
prior to survey or other competing age composition information the
values were left at relatively low values to reflect the uncertainty of
the early period age composition information. The sample sizes for the
start and final model are shown in Table 1.24.

\#\#Parameters estimated outside of the assessment model

\#\#\#Natural mortality and maturity at age j For all models, fixed
natural mortality rates at age were assumed (M=0.9, 0.45, and 0.3 for
ages 1, 2, and 3+ respectively; Wespestad and Terry 1984). These values
have been applied to catch-age models and forecasts since 1982 and
appear reasonable for pollock. When predation was explicitly considered
estimates tend to be higher and more variable (Holsman et al.~2015;
Livingston and Methot 1998; Hollowed et al.~2000). Clark (1999) noted
that specifying a conservative (lower) natural mortality rate may be
advisable when natural mortality rates are uncertain. In the 2014
assessment different natural mortality vectors were evaluated in which
the ``Lorenzen'' approach and that of Gislason et al (2010) were tested.
The values assumed for pollock natural mortality-at-age and
maturity-at-age (for all models; Smith 1981) consistent with previous
assessments were:

Age 1 2 3 4 5 6 7 8 9 10 11 12 13 14 15 Model 1.0 M 0.900 0.450 0.300
0.300 0.300 0.300 0.300 0.300 0.300 0.300 0.300 0.300 0.300 0.300 0.300
Prop. Mature 0.000 0.008 0.290 0.642 0.842 0.902 0.948 0.964 0.970 1.000
1.000 1.000 1.000 1.000 1.000

In the supplemental multi-species assessment model alternative values of
age and time-varying natural mortality are presented. Those estimates
indicate higher values than used here. As a sensitivity, a profile of
different fixed age 3+ values of natural mortality showed that given the
assessment model configuration outlined below (for Model 16.1) survey
age compositions favored lower values of M while the fishery age
composition favored higher values (Fig. 1.17). This is somewhat
unsurprising since in recent years the BTS data show increased
abundances of ``fully selected'' cohorts. Hence, given the model
specification (asymptotic selectivity for the BTS age composition data),
lower natural mortality rates would be consistent with those data. Given
these trade-offs, structural model assumptions were held to be the same
as previous years for consistency (i.e., the mortality schedule
presented above).

Maturity-at-age values used for the EBS pollock assessment are
originally based on Smith (1981) and have been reevaluated (e.g., Stahl
2004; Stahl and Kruse 2008a; and Ianelli et al.~2005). These studies
found inter-annual variability but general consistency with the current
assumed schedule of proportion mature at age. Trends in roe production
suggest some possible differences in the warm conditions observed in
2016 and current research is underway to evaluate potential consequences
(S. Neidetcher AFSC, pers. Comm.).

\#\#\#Length and Weight at Age Age determination methods have been
validated for pollock (Kimura et al.~1992; Kimura et al.~2006, and
Kastelle and Kimura 2006). EBS pollock size-at-age show important
differences in growth with differences by sex, area, year, and year
class. Pollock in the northwest area are typically smaller at age than
pollock in the southeast area. The differences in average weight-at-age
are taken into account by stratifying estimates of catch-at-age by year,
area, season, and weighting estimates proportional to catch.

The assessment model for EBS pollock accounts for numbers of individuals
in the population. As noted above, management recommendations are based
on allowable catch levels expressed as tons of fish. While estimates of
pollock catch-at-age are based on large data sets, the data are only
available up until the most recent completed calendar year of fishing
(e.g., 2015 for the assessment conducted in 2016). Consequently,
estimates of weight-at-age in the current year are required to map total
catch biomass (typically equal to the quota) to numbers of fish caught
(in the current year). Therefore, these estimates can have large impacts
on recommendations (e.g., ABC and OFL).

The mean weight at age in the fishery can vary due to environmental
conditions in addition to spatial and temporal patterns of the fishery.
Bootstrap distributions of the within-year sampling variability indicate
it is relatively small compared to between-year variability in mean
weights-at-age. This implies that processes determining mean weights in
the fishery cause more variability than sampling (Table 1.25). The
coefficients of variation between years are on the order of 6\% to 9\%
(for the ages that are targeted) whereas the sampling variability is
generally around 1\% or 2\%.

The approach to account for the identified mean weight-at-age having
clear year and cohort effects was refined due to comments from the Plan
Team, CIE and SSC. For details of this approach (presented in September
and October to the Plan Team and SSC) refer to appendix 1A of this
chapter. Results of this method show the relative variability between
years and cohorts and provide estimates (and uncertainty) for 2016-2018
(Fig. 1.18; Table 1.25).

\#\#Parameters estimated within the assessment model For the selected
model, 929 parameters were estimated conditioned on data and model
assumptions. Initial age composition, subsequent recruitment, and
stock-recruitment parameters account for 76 parameters. This includes
vectors describing the initial age composition (and deviation from the
equilibrium expectation) in the first year (as ages 2-15 in 1964) and
the recruitment mean and deviations (at age 1) from 1964-2016 and
projected recruitment variability (using the variance of past
recruitments) for five years (2016-2021). The two-parameter
stock-recruitment curve is included in addition to a term that allows
the average recruitment before 1964 (that comprises the initial age
composition in that year) to have a mean value different from subsequent
years. Note that the stock-recruit relationship is fit only to stock and
recruitment estimates from 1978 year-class through to the 2013
year-class.

Fishing mortality is parameterized to be semi-separable with year and
age (selectivity) components. The age component is allowed to vary over
time; changes are allowed in each year. The mean value of the age
component is constrained to equal one and the last 5 age groups (ages
11-15) are specified to be equal. This latter specification feature is
intended to reduce the number of parameters while acknowledging that
pollock in this age-range are likely to exhibit similar life-history
characteristics (i.e., unlikely to change their relatively availability
to the fishery with age). The annual components of fishing mortality
result in 54 parameters and the age-time selectivity schedule forms a
10x53 matrix of 530 parameters bringing the total fishing mortality
parameters to 584.

Selectivity-at-age estimates for the bottom trawl survey are specified
with age and year specific deviations in the average selectivity-at-age.
For the AT survey, which began in 1979, parameters are used to specify
age-time specific availability. Time-varying survey selectivity is
estimated to account for the changes in availability of pollock to the
survey gear and is constrained by pre-specified variance terms. Five
catchability coefficients were estimated: one each for the early fishery
catch-per-unit effort (CPUE) data (from Low and Ikeda, 1980), the early
bottom trawl survey data (where only 6 strata were surveyed), the main
bottom trawl survey data (including all strata surveyed), the AT survey
data, and the AVO data. No prior distribution was used for any of the
indices. The selectivity parameters for the 2 main indices total 132
(the CPUE and AVO data mirror the fishery and AT survey selectivities,
respectively).

Additional fishing mortality rates used for recommending harvest levels
are estimated conditionally on other outputs from the model. For
example, the values corresponding to the F40\%, F35\% and FMSY harvest
rates are found by satisfying the constraint that, given age-specific
population parameters (e.g., selectivity, maturity, mortality,
weight-at-age), unique values exist that correspond to these fishing
mortality rates. The likelihood components that are used to fit the
model can be categorized as:

\begin{itemize}
\tightlist
\item
  Total catch biomass (log-normal, \(\sigma=0.05\))\\
\item
  Log-normal indices of pollock biomass; bottom trawl surveys assume
  annual estimates of sampling error, as represented in Fig. 1.9; for
  the AT index the annual errors were specified to have a mean of 0.20;
  while for the AVO data, a value relative to the AT index was estimated
  and gave a mean of about 0.32).\\
\item
  Fishery and survey proportions-at-age estimates (robust
  quasi-multinomial with effective sample sizes presented in Table
  1.24).\\
\item
  Age 1 index from the AT survey (CV set equal to 30\% as in prior
  assessments).\\
\item
  Selectivity constraints: penalties/priors on age-age variability, time
  changes, and decreasing (with age) patterns.\\
\item
  Stock-recruitment: penalties/priors involved with fitting a stochastic
  stock-recruitment relationship within the integrated model.\\
\item
  ``Fixed effects'' terms accounting for cohort and year sources of
  variability in fishery mean weights-at-age estimated based on
  available data from 1991-2015 and externally estimated variance terms
  as described in Appendix 1A.
\end{itemize}

Work evaluating temperature and predation-dependent effects on the
stock-recruitment estimates has begun (Spencer et al.~2016). His
approach modified the estimation the of the stock-recruitment
relationship by including the effect of temperature and predation
mortality. A relationship between recruitment residuals and temperature
was noted (similar to that found in Mueter et al., 2011) and lower
pollock recruitment during warmer conditions might be expected. Similar
results relating summer temperature conditions to subsequent pollock
recruitment for recent years were also found by Yasumiishi et
al.~(2015). The extent that such relationships affect the
stock-recruitment estimates (and future productivity) is a continuing
area of research.

\#Results \#\#Model evaluation Incremental updates and additions of new
data to the model 15.1 accepted last year suggests that most of the
changes in results are due to the data added rather than the
modifications to tuning to biomass versus numbers and to the re-tuning
adjustments for sample size estimates (Fig. 1.19). Subsequent model
evaluations and sensitivities were focused on assumptions relative to
projections (average weight, selectivity, and stock recruitment
estimates) and these had little or no bearing on fitting historical
data. For Model 16.1, four sub-models were run to show the effect of
adding data to the model this year. The addition of age composition data
from the fishery and different surveys shows that the proportion of
3-year old pollock in the 2015 fishery was much higher than expected
whereas that same year class (2012) was slightly less than expected in
the BTS data (Fig. 1.20). A similar effect can be observed in the
incremental fitting of new data for the AT and BTS time series (Fig.
1.21). In particular, the BTS biomass estimate reduces the upward trend
predicted when those data are excluded. As part of the sample size
re-weighting process, a diagnostic for evaluating Francis weight
performance compares observed versus model predicted mean age by
different composition datasets. The fits for Model 16.1 appear to be
reasonable (Fig. 1.22) and compare favorably with Model 15.1 (Table
1.26). However, comparisons between these models are difficult based on
goodness of fit alone since different indices are used for tuing and
statistical weights for the for composition data differ.

Relative to the average weights-at-age projected for the fishery and
alternative assumptions about how to estimate ``future selectivity''
Ianelli et al.~(2015) showed how the buffer between ABC and OFL
(computed as 1-ABC/OFL) for Tier 1 varies as well as the relative value
of the maximum permissible ABC. The uncertainty in future mean
weights-at-age had a relatively large impact and the selectivity
estimation (based on the number of recent years over which to average
selectivity) also affected variability in results.

The estimated parameters and standard errors are provided in Table 1.27
and summary model results are given in (Table 1.28). The code for the
model (with dimensions and links to parameter names) and input files are
available upon request to the lead author.

The estimated selectivity pattern changes over time and reflects to some
degree the extent to which the fishery is focused on particularly
prominent year-classes (Fig. 1.23). The model fits the fishery
age-composition data quite well under this form of selectivity (Fig.
1.24). The fit to the early Japanese fishery CPUE data (Low and Ikeda
1980) is consistent with the population trends for this period (Fig.
1.25). The fit to the fishery-independent index from the 2006-2015 AVO
data shows a slightly declining rather than increasing trend to 2015
(Fig. 1.26).

Bottom-trawl survey selectivity and fits to the numbers of age 2 and
older pollock indicate that the model predicts fewer pollock than
observed in the 2014 and 2015 survey but slightly more than observed in
the 2012, 2013 and 2016 surveys even though the model is tuned to
biomass rather than numbers as depicted in Fig. 1.27). The pattern of
bottom trawl survey age composition data in recent years shows a decline
in the abundance of older pollock since 2011. The 2006 year-class
observations are below model expectations in 2012 and 2013, partly due
to the fact that in 2010 the survey estimates are greater than the model
predictions (Fig. 1.28).

The AT survey selectivity estimates could differ in the 1979 survey;
(Fig. 1.29; top panel). The fit to the numbers of age 2 and older
pollock in the AT survey generally falls within the confidence bounds of
the survey sampling distributions (here assumed to have an average CV of
20\%) with a reasonable pattern of residuals (Fig. 1.29, bottom panel).
The AT age compositions consistently track large year classes through
the population and the model fits these patterns reasonably well (Fig.
1.30).

\#\#Time series results The time series of begin-year biomass estimates
(ages 3 and older) suggests that the abundance of Eastern Bering Sea
pollock remained at a high level from 1981-88, with estimates ranging
from 8 to 12 million t (Table 1.29). Historically, biomass levels
increased from 1979 to the mid-1980s due to the strong 1978 and
relatively strong 1982 and 1984 year classes recruiting to the fishable
population. The stock is characterized by peaks in the mid-1980s, the
mid-1990s and again appears to be increasing to new highs over 13
million t following the low in 2008 of 4.9 million t.

The level of fishing relative to biomass estimates show that the
spawning exploitation rate (SER, defined as the percent removal of egg
production in each spawning year) has been mostly below 20\% since 1980
(Fig. 1.31). During 2006 and 2007 the rate averaged more than 20\% and
the average fishing mortality for ages 3-8 increased during the period
of stock decline. The estimate for 2009 through 2016 was below 20\% due
to the reductions in TACs relative to the maximum permissible ABC values
and increased in the spawning biomass. The average F (ages 3-8)
increased in 2011 to above 0.25 when the TAC increased but has dropped
since then and in 2016 is estimated at about 0.16. Age specific fishing
mortality rates reflect these patterns and show some increases in the
oldest ages from 2011-2013 but also indicate a decline in recent years
(Fig. 1.32). The estimates of age 3+ pollock biomass were mostly higher
than the estimates from previous years (Fig. 1.33, Table 1.29).

To evaluate past management and assessment performance it can be useful
to plot estimated fishing mortality relative to some reference values.
For EBS pollock, we computed the reference fishing mortality from Tier 1
(unadjusted) and calculated the historical values for FMSY (since
selectivity has changed over time). Since 1977 the current estimates of
fishing mortality suggest that during the early period, harvest rates
were above FMSY until about 1980. Since that time, the levels of fishing
mortality have averaged about 35\% of the FMSY level (Fig. 1.34).

\#\#Recruitment

Model estimates indicate that both the 2008 and 2012 year classes are
well above the average level (Fig. 1.35). The stock-recruitment curve as
fit within the integrated model shows a fair amount of variability both
in the estimated recruitments and in the uncertainty of the curve (Fig.
1.36). Note that the 2014 and 2015 year classes (as age 1 recruits in
2015 and 2016) are excluded from the stock-recruitment curve estimation.
Separate from fitting the stock-recruit relationship within the model,
examining the estimated recruits-per-spawning biomass shows variability
over time but seems to lack trend and also is consistent with the Ricker
stock-recruit relationship used within the model (Fig. 1.37).

Environmental factors affecting recruitment are considered important and
contribute to the variability. Previous studies linked strong Bering Sea
pollock recruitment to years with warm sea temperatures and northward
transport of pollock eggs and larvae (Wespestad et al.~2000; Mueter et
al.~2006). As part of the Bering-Aleutian Salmon International Survey
(BASIS) project research has also been directed toward the relative
density and quality (in terms of condition for survival) of
young-of-year pollock. For example, Moss et al.~(2009) found age-0
pollock were very abundant and widely distributed to the north and east
on the Bering Sea shelf during 2004 and 2005 (warm sea temperature; high
water column stratification) indicating high northern transport of
pollock eggs and larvae during those years. More recently, Mueter et
al.~(2011) found that warmer conditions tended to result in lower
pollock recruitment in the EBS. This is consistent with the hypothesis
that when sea temperatures on the eastern Bering Sea shelf are warm and
the water column is highly stratified during summer, age-0 pollock
appear to allocate more energy to growth than to lipid storage
(presumably due to a higher metabolic rate), leading to low energy
density prior to winter. This then may result in increased over-winter
mortality (Swartzman et al.~2005, Winter et al.~2005). Ianelli et
al.~(2011) evaluated the consequences of current harvest policies in the
face of warmer conditions with the link to potentially lower pollock
recruitment and noted that the current management system is likely to
face higher chances of ABCs below the historical average catches.

Considering the factors affecting recruitment, including the probability
that stationarity in the stock-recruit relationship is unlikely, a
subjective approach to accounting for additional uncertainty was
developed. As a first step, and failing development of a comprehensive
ensemble of models which could somehow be more objective, two
alternatives to the base-case stock-recruit relationship scenarios were
included: one that reduced the influence of the internal model estimates
of stock and recruitment in specifying the stock-recruit relationship
(so-called ``low conditioned'' model) and a second one that was
intermediate to the base-case scenario and the low conditioned option.
For illustration, the 3 cases are shown in two panels (Fig. 1.38. The
1-ABC/OFL buffer for the cases result in: 17\%, 14\%, and 12\%,
respectively. Also the values for steepness (and hence point estimates
of Fmsy) change in these scenarios (0.568, 0.618, and 0.685,
respectively). In lieu of eliciting a suite of models to capture
structural uncertainty, the moderate condition specification was
selected for ABC/OFL recommendations. Future research will attempt to
more fully support and characterize the range applicable.

\#\#Retrospective analysis

Model 16.1, as with past model evaluations, indicate retrospective
sensitivity to data available (Fig. 1.39). On balance, for 10 years of
retrospective analysis, even though the variability was high, the
average bias was low with Mohn's rho near zero (-0.004).

\#Harvest recommendations

The estimate of BMSY is 2,165,000 t (with a CV of 20\%) which is less
than the projected 2017 spawning biomass of 4,600,000 t; Table 1.29).
For 2016, the Tier 1 levels of yield are 3,120,000 t from a fishable
biomass estimated at around 7,830,000 t (Table 1.30). Estimated
numbers-at-age are presented in Table 1.31 and estimated catch-at-age is
presented in Table 1.32. Estimated summary biomass (age 3+), female
spawning biomass, and age-1 recruitment are given in Table 1.33.

Model results indicate that spawning biomass will be above B40\%
(2,643,000 t) in 2017 and about 212\% of the BMSY level. The probability
that the current stock size is below 20\% of B0 (based on estimation
uncertainty alone) is \textless{}0.1\% for 2016 and 2017.

A diagnostic (see Eq. 14 in appendix) on the impact of fishing shows
that the 2016 spawning stock size is about 66\% of the predicted value
had no fishing occurred since 1978 (Table 1.29). This compares with the
62\% of B100\% (based on the SPR expansion using mean recruitment from
1978-2012) and 71\% of B0 (based on the estimated stock-recruitment
curve). The latter two values are based on expected recruitment from the
mean value since 1978 or from the estimated stock recruitment
relationship.

\#\#Amendment 56 Reference Points Amendment 56 to the BSAI Groundfish
Fishery Management Plan (FMP) defines overfishing level (OFL), the
fishing mortality rate used to set OFL (FOFL), the maximum permissible
ABC, and the fishing mortality rate used to set the maximum permissible
ABC. The fishing mortality rate used to set ABC (FABC) may be less than
this maximum permissible level, but not greater. Estimates of reference
points related to maximum sustainable yield (MSY) are currently
available. However, their reliability is questionable. We therefore
present both reference points for pollock in the BSAI to retain the
option for classification in either Tier 1 or Tier 3 of Amendment 56.
These Tiers require reference point estimates for biomass level
determinations. Consistent with other groundfish stocks, the following
values are based on recruitment estimates from post-1976 spawning
events:

\(B_{MSY}\) = 2,165 thousand t female spawning biomass

\(B_{0}\) = 5,700 thousand t female spawning biomass

\(B_{100\%}\) = 6,608 thousand t female spawning biomass

\(B_{40\%}\) = 2,643 thousand t female spawning biomass

\(B_{35\%}\) = 2,313 thousand t female spawning biomass

\#\#Specification of OFL and Maximum Permissible ABC

Assuming the moderately diffuse stock-recruit relationship the 2017
spawning biomass is estimated to be 4,600,000 t (at the time of
spawning, assuming the stock is fished at recommended ABC level). This
is above the BMSY value of 2,165,000 t. Under Amendment 56, this stock
has qualified under Tier 1 and the harmonic mean value is considered a
risk-averse policy since reliable estimates of FMSY and its pdf are
available (Thompson 1996). The exploitation-rate type value that
corresponds to the FMSY level was applied to the fishable biomass for
computing ABC levels. For a future year, the fishable biomass is defined
as the sum over ages of predicted begin-year numbers multiplied by age
specific fishery selectivity (normalized to the value at age 6) and mean
body mass.

Since the 2017 female spawning biomass is estimated to be above the BMSY
level (2,165,000 t) and the B40\% value (2,643,000 t) in 2017 and if the
2016 catch equals 1.35 million t, the OFL and maximum permissible ABC
values by the different Tiers would be:

\begin{longtable}[]{@{}ccrrr@{}}
\toprule
Tier & Year & & MaxABC & OFL\tabularnewline
\midrule
\endhead
1a & 2018 & & 3,120,000 t & 3,640,000 t\tabularnewline
1a & 2019 & & 3,740,000 t & 4,360,000 t\tabularnewline
& & & &\tabularnewline
Tier & Year & & MaxABC & OFL\tabularnewline
3a & 2018 & & 2,800,000 t & 2,970,000 t\tabularnewline
3a & 2019 & & 2,979,000 t & 3,430,000 t\tabularnewline
\bottomrule
\end{longtable}

\begin{longtable}[]{@{}ccccc@{}}
\toprule
Tier & Year & & MaxABC & OFL\tabularnewline
\midrule
\endhead
1a & 2017 & & 3,120,000 t & 3,640,000 t\tabularnewline
1a & 2018 & & 3,740,000 t & 4,360,000 t\tabularnewline
& & & &\tabularnewline
Tier & Year & & MaxABC & OFL\tabularnewline
3a & 2017 & & 2,800,000 t & 2,970,000 t\tabularnewline
3a & 2018 & & 2,979,000 t & 3,430,000 t\tabularnewline
\bottomrule
\end{longtable}

\#\#Standard Harvest Scenarios and Projection Methodology

A standard set of projections is required for each stock managed under
Tiers 1, 2, or 3 of Amendment 56. This set of projections encompasses
seven harvest scenarios designed to satisfy the requirements of
Amendment 56, the National Environmental Policy Act, and the
Magnuson-Stevens Fishery Conservation and Management Act (MSFCMA). While
EBS pollock is generally considered to fall within Tier 1, the standard
projection model requires knowledge of future uncertainty in FMSY. Since
this would require a number of additional assumptions that presume
future knowledge about stock-recruit uncertainty, the projections in
this subsection are based on Tier 3.

For each scenario, the projections begin with the vector of 2016 numbers
at age estimated in the assessment. This vector is then projected
forward to the beginning of 2017 using the schedules of natural
mortality and selectivity described in the assessment and the best
available estimate of total (year-end) catch assumed for 2016. In each
subsequent year, the fishing mortality rate is prescribed on the basis
of the spawning biomass in that year and the respective harvest
scenario. Annual recruitments are simulated from an inverse Gaussian
distribution whose parameters consist of maximum likelihood estimates
determined from recruitments estimated in the assessment. Spawning
biomass is computed in each year based on the time of peak spawning and
the maturity and weight schedules described in the assessment. Total
catch is assumed to equal the catch associated with the respective
harvest scenario in all years. This projection scheme is run 1,000 times
to obtain distributions of possible future stock sizes and catches under
alternative fishing mortality rate scenarios.

Five of the seven standard scenarios will be used in an Environmental
Assessment prepared in conjunction with the final SAFE. These five
scenarios, which are designed to provide a range of harvest alternatives
that are likely to bracket the final TAC for 2017 and 2018, are as
follows (max FABC refers to the maximum permissible value of FABC under
Amendment 56):

\begin{description}
\item[Scenario 1:]   
In all future years, F is set equal to max FABC. (Rationale:  Historically, TAC has been constrained by ABC, so this scenario provides a likely upper limit on future TACs).
\item[Scenario 2:]   
In 2019 the catch is set equal to 1.35 million t and in future years F is set equal to the Tier 3 estimate (Rationale: this was estimated to be the level of catch where the spawning biomass in 2016 would equal the 2014 estimate). 
\item[Scenario 3:]   
In all future years, F is set equal to the 2012-2016 average F. (Rationale:  For some stocks, TAC can be well below ABC, and recent average F may provide a better indicator of FTAC than FABC.)
\item[Scenario 4:]   
Scenario 4: In all future years, F is set equal to F60%. (Rationale:  This scenario provides a likely lower bound on FABC that still allows future harvest rates to be adjusted downward when stocks fall below reference levels. This was requested by public comment for the DSEIS developed in 2006)
\item[Scenario 5:]   
Scenario 5: In all future years, F is set equal to zero. (Rationale:  In extreme cases, TAC may be set at a level close to zero.)
\item[Scenario 6:]   
In all future years, F is set equal to FOFL. (Rationale:  This scenario determines whether a stock is overfished. If the stock is expected to be 1) above its MSY level in 2016 or 2) above ½ of its MSY level in 2016 and above its MSY level in 2026 under this scenario, then the stock is not overfished.)
\item[Scenario 7:]   
In 2017 and 2018, F is set equal to max FABC, and in all subsequent years, F is set equal to FOFL. (Rationale:  This scenario determines whether a stock is approaching an overfished condition. If the stock is 1) above its MSY level in 2018 or 2) above 1/2 of its MSY level in 2018 and expected to be above its MSY level in 2028 under this scenario, then the stock is not approaching an overfished condition). 
\end{description}

The latter two scenarios are needed to satisfy the MSFCMA's requirement
to determine whether a stock is currently in an overfished condition or
is approaching an overfished condition. These two scenarios are as
follow (for Tier 3 stocks, the MSY level is defined as B35\%):

\#Projections and status determination For the purposes of these
projections, we present results based on selecting the F40\% harvest
rate as the max FABC value and use F35\% as a proxy for FMSY. Scenarios
1 through 7 were projected 14 years from 2016 (Table 1.34). Under the
maximum permissible catch level in Tier 3, the expected spawning biomass
will decline until 2020 and stabilize slightly above B40\% (in
expectation; Fig. 1.40).

Any stock that is below its minimum stock size threshold (MSST) is
defined to be overfished. Any stock that is expected to fall below its
MSST in the next two years is defined to be approaching an overfished
condition. Harvest scenarios 6 and 7 are used in these determinations as
follows:

Is the stock overfished? This depends on the stock's estimated spawning
biomass in 2016:

\begin{itemize}
\tightlist
\item
  If spawning biomass for 2016 is estimated to be below ½ \(B_{35\%}\)
  the stock is below its MSST.\\
\item
  If spawning biomass for 2016 is estimated to be above B35\%, the stock
  is above its MSST.\\
\item
  If spawning biomass for 2016 is estimated to be above ½ B35\% but
  below B35\%, the stock's status relative to MSST is determined by
  referring to harvest scenario 6 (Table 1.34). If the mean spawning
  biomass for 2026 is below B35\%, the stock is below its MSST.
  Otherwise, the stock is above its MSST.
\end{itemize}

Is the stock approaching an overfished condition? This is determined by
referring to harvest Scenario 7:

\begin{itemize}
\tightlist
\item
  If the mean spawning biomass for 2018 is below ½ B35\%, the stock is
  approaching an overfished condition.\\
\item
  If the mean spawning biomass for 2018 is above B35\%, the stock is not
  approaching an overfished condition.\\
\item
  If the mean spawning biomass for 2018 is above ½ B35\% but below
  B35\%, the determination depends on the mean spawning biomass for
  2028. If the mean spawning biomass for 2028 is below B35\%, the stock
  is approaching an overfished condition. Otherwise, the stock is not
  approaching an overfished condition.
\end{itemize}

For scenarios 6 and 7, we conclude that pollock is not below MSST for
the year 2016, nor is it expected to be approaching an overfished
condition based on Scenario 7 (the mean spawning biomass in 2016 is
above the \(B_{35\%}\) level; Table 1.34). Tier 1 calculations for ABC
and OFL values in 2017 and 2018 (assuming catch is 1,350,000 t in 2017
are given in Table 1.35. Based on this, the EBS pollock stock is not
being subjected to overfishing, is not overfished, and not approaching a
condition of being overfished

\#\#ABC Recommendation ABC levels are affected by estimates of FMSY
(which depends principally on the stock-recruitment relationship and
demographic schedules such as selectivity-at-age, maturity, growth), the
BMSY level, and current stock size (both spawning and fishable). Updated
data and analysis result in an estimate of 2016 spawning biomass (4,070
kt) that is about 212\% of BMSY (2,165 kt). The replacement
yield---defined as the catch next year that is expected to achieve a
2018 spawning biomass estimate equal to that from 2016---is estimated to
be about 2,500,000 t.

The EBS pollock stock appears to have rebounded from the 2008 low point
and shows significant increases due to two strong year classes (2008 and
2012). However, there remain several concerns about the medium-term
stock conditions. Namely,

\begin{enumerate}
\def\labelenumi{\arabic{enumi}.}
\item
  The conditions in summer 2016 were the warmest recorded over the
  period 1982-2016; additional precaution may be warranted since warm
  conditions are thought to negatively affect the survival of larval and
  juvenile pollock.\\
\item
  The acoustic survey found very few one-year-old pollock in summer 2016
  (the BTS data show about average 1-year olds).
\item
  The current BTS data show low abundances of pollock aged 10 and older.
  Historically there had been good representation of older fish in data
  from this survey. This is somewhat expected given the poor
  year-classes observed during the period 2000-2005.
\item
  The BTS showed patchier concentrations of pollock compared to recent
  years. This can result in increased uncertainty in the estimates. This
  patchier distribution may also reflect somewhat better nominal fishery
  catch rates.
\item
  The multispecies model suggests that the BMSY level is around 3.6
  million t instead of the \textasciitilde{}2 million t estimated in the
  current assessment (noting that the total natural mortality is higher
  in the multispecies model).
\item
  Roe production has dropped in 2015 in the B-season. Recent data show
  that \textasciitilde{}15\% of annual roe production has occurred from
  June-October whereas in 2015 and 2016 the production is
  \textasciitilde{}5\%.
\item
  The selection of a single model, though attempting to account for
  uncertainties due to process errors, ignores structural uncertainty in
  model specification. Including such structural uncertainties may
  reflect the type of variability in stock-recruit relationship depicted
  in the scenario where conditioning the curve on the assessment results
  is lowered.
\item
  The euphausiid index (see Ecosystem considerations, this volume)
  decreased from the 2014 estimates and has declined since the 2009
  peak. This may negatively affect survival rates of juvenile pollock
  prior to recruiting to the fishery.
\item
  Pollock are an important prey species for the ecosystem; there's been
  a 12\% decline in St.~Paul Island pup production from 2014-2016 which,
  when combined information on the other fur seal population components
  (Bogoslof and St.~George Islands), indicates an estimated 2.5\%
  decline in the overall Eastern Stock fur seal population. Maintaining
  prey availability may provide better foraging opportunities for the
  fur seal stock to minimize further declines.
\item
  Whilst outside of ABC considerations, it seems that maintaining the
  stock at relatively high levels and achieving fishery catch rates
  observed in 2016 B-season may help to minimize Chinook salmon bycatch
  (noting that the total effort required to catch 600 kt in the 5 most
  recent B-seasons was substantially smaller this year)
\end{enumerate}

Given these factors, a 2017 ABC of 2,800,000 t is recommended based on
the Tier 3 estimates as conservatively selected by the SSC in 2014 and
2015. We recognize that the actual catch will be constrained by other
factors (the 2 million t OY BSAI groundfish catch limit; bycatch
avoidance measures). The alternative maximum permissible Tier 1a ABC
seems clearly risky. Such high catches would result in unprecedented
variability and removals from the stock (and considerably more capacity
and effort). Adopting a more stable catch system would also result in
less spawning stock variability.

\#Ecosystem considerations In general, a number of key issues for
ecosystem conservation and management can be highlighted. These include:

\begin{itemize}
\item
  Preventing overfishing;
\item
  Avoiding habitat degradation;
\item
  Minimizing incidental bycatch;
\item
  Monitoring bycatch and the level of discards; and
\item
  Considering multi-species trophic interactions relative to harvest
  policies.
\end{itemize}

For the case of pollock in the Eastern Bering Sea, the NPFMC and NMFS
continue to manage the fishery on the basis of these issues in addition
to the single-species harvest approach (Hollowed et al.~2011). The
prevention of overfishing is clearly set out as the main guideline for
management. Habitat degradation has been minimized in the pollock
fishery by converting the industry to pelagic-gear only. Bycatch in the
pollock fleet is closely monitored by the NMFS observer program and
managed on that basis. Discard rates of many species have been reduced
in this fishery and efforts to minimize bycatch continue.

In comparisons of the Western Bering Sea (WBS) with the Eastern Bering
Sea using mass-balance food-web models based on 1980-85 summer diet
data, Aydin et al.~(2002) found that the production in these two systems
is quite different. On a per-unit-area measure, the western Bering Sea
has higher productivity than the EBS. Also, the pathways of this
productivity are different with much of the energy flowing through
epifaunal species (e.g., sea urchins and brittlestars) in the WBS
whereas for the EBS, crab and flatfish species play a similar role. In
both regions, the keystone species in 1980-85 were pollock and Pacific
cod. This study showed that the food web estimated for the EBS ecosystem
appears to be relatively mature due to the large number of
interconnections among species. In a more recent study based on 1990-93
diet data (see Appendix 1 of the Ecosystem Considerations chapter for
methods), pollock remain in a central role in the ecosystem. The diet of
pollock is similar between adults and juveniles with the exception that
adults become more piscivorous (with consumption of pollock by adult
pollock representing their third largest prey item).

Regarding specific small-scale ecosystems of the EBS, Ciannelli et
al.~(2004a, 2004b) presented an application of an ecosystem model scaled
to data available around the Pribilof Islands region. They applied
bioenergetics and foraging theory to characterize the spatial extent of
this ecosystem. They compared energy balance, from a food web model
relevant to the foraging range of northern fur seals and found that a
range of 100 nautical mile radius encloses the area of highest energy
balance representing about 50\% of the observed foraging range for
lactating fur seals. This has led to a hypothesis that fur seals depend
on areas outside the energetic balance region. This study develops a
method for evaluating the shape and extent of a key ecosystem in the EBS
(i.e., the Pribilof Islands). Furthermore, the overlap of the pollock
fishery and northern fur seal foraging habitat (see Sterling and Ream
2004, Zeppelin and Ream 2006) will require careful monitoring and
evaluation.

A brief summary of these two perspectives (ecosystem effects on pollock
stock and pollock fishery effects on ecosystem) is given in Table 1.36.
Unlike the food-web models discussed above, examining predators and prey
in isolation may overly simplify relationships. This table serves to
highlight the main connections and the status of our understanding or
lack thereof.

\hypertarget{ecosystem-effects-on-the-ebs-pollock-stock}{%
\subsection{Ecosystem effects on the EBS pollock
stock}\label{ecosystem-effects-on-the-ebs-pollock-stock}}

The pollock stock condition appears to have benefitted substantially
from the recent conditions in the EBS. The conditions on the shelf
during 2008 apparently affected conditions for age-0 northern rock sole
due to cold conditions and apparently unfavorable currents that retain
them into the over-summer nursery areas (Cooper et al.~2014). It may be
that such conditions favor pollock recruitment. Hollowed et al.~(2012)
provided an extensive review of habitat and density for age-0 and age-1
pollock based on extensive survey data. They noted that during cold
years, age-0 pollock were distributed primarily in the outer domain in
waters greater than 1ºC and during warm years, age-0 pollock were
distributed mostly in the middle domain. This temperature relationship,
along with interactions with available food in early-life stages,
appears to have important implications for pollock recruitment success
(Coyle et al.~2011). The fact that the 2012 year-class, while uncertain,
appears to be also high creating a favorable stock trend in the near
term.

A separate section presented this year updates multispecies model with
more recent data and is presented as a supplement to the BSAI SAFE
report. In this approach, a number of simplifications for the individual
species data and fisheries processes (e.g., no time varying selectivity
in the fishery and only design-based survey indices). However, that
model mimics the pattern and abundances with the single species
reasonably well. It also allows specific questions to be addressed
regarding pollock TACs. For example, since predation (and cannibalism)
is explicitly modeled, the impact of relative stock sizes on subsequent
recruitment to the fishery can be now be directly estimated and
evaluated (in the model presented here, cannibalism is explicitly
accounted for in the assumed Ricker stock-recruit relationship).

Euphausiids, principally Thysanoessa inermis and T. raschii, are among
the most important prey items for pollock in the Bering Sea (Livingston,
1991; Lang et al., 2000; Brodeur et al., 2002; Cianelli et al., 2004;
Lang et al., 2005). Buckley et al.~(2016) showed spatial patterns of
pollock foraging by size of predators. For example, the northern part of
the outer domain (closest to the shelf break) tends to be more
piscivorous than counterparts in other areas (Fig. 1.41). This figure
also shows that euphausiids make up a larger component of the diet in
the southern areas. The euphausiid abundance on the Bering Sea shelf is
presented as a section of the 2016 Ecosystem Considerations Chapter of
the SAFE report and shows a continued decline in abudance since the peak
in 2009 (for details see De Robertis et al.~(2010) and Ressler et
al.~(2012). The role that the apparent recent 2009 peak abundance had in
the survival of the 2008 year class of EBS pollock is interesting.
Contrasting this with how the feeding ecology of the 2012 year class
(also apparently well above average) may differ is something to evaluate
in the future.

\#\#EBS pollock fishery effects on the ecosystem.

Since the pollock fishery is primarily pelagic in nature, the bycatch of
non-target species is small relative to the magnitude of the fishery
(Table 1.37). Jellyfish represent the largest component of the bycatch
of non-target species and had averaged around 5-6 thousand tons per year
but more than doubled in 2014 but has dropped in 2015. The data on
non-target species shows a high degree of inter-annual variability,
which reflects the spatial variability of the fishery and high
observation error. This variability may reduce the ability to detect
significant trends for bycatch species.

The catch of other target species in the pollock fishery represent less
than 1\% of the total pollock catch. Incidental catch of Pacific cod has
increased since 1999 but remains below the 1997 levels (Table 1.38). The
incidental catch of flatfish was variable over time and has increased,
particularly for yellowfin sole. Proportionately, the incidental catch
has decreased since the overall levels of pollock catch have increased.
In fact, the bycatch of pollock in other target fisheries is more than
double the bycatch of target species in the pollock fishery (Table
1.39).

A high number of non-Chinook salmon (nearly all made up of chum salmon)
was observed in 2014 and 2015 (about 13\% above the 2003-2013 average)
after the low level observed in 2012 (Table 1.40). Chinook salmon
bycatch in 2015 was 54\% of the 2003-2015 mean value consistent with the
magnitude of bycatch since the implementation of Amendment 91 in 2011.
Ianelli and Stram (2014) provide estimates of the bycatch impact on
Chinook salmon runs to the coastal west Alaska region and found that the
peak bycatch levels exceeded 7\% of the total run return. Since 2011,
the impact has been estimated to be below 2\%.

\#Data gaps and research priorities

The available data for EBS pollock are extensive yet many processes
behind the observed patterns are poorly understood. For example, the
recent bottom trawl surveys found abundance levels for the 2008 and now
2012 year class appear to be estimated at high levels. Research on
developing and testing plausible hypotheses about the underlying
processes that cause such observations is needed. This should include
examining potential effects of temporal changes in survey stations and
using spatial processes for estimation purposes (e.g., combining
acoustic and bottom trawl survey data). The application of the
geostatistical methods (presented for comparative purposes above) seem
like a reasonable approach to statistically model disparate data sources
for generating better abundance indices.

More studies on spatial dynamics, including the relationship between
climate and recruitment and trophic interactions of pollock within the
ecosystem would be useful for improving ways to evaluate the current and
alternative fishery management system. In particular, studies
investigating the processes affecting recruitment of pollock in the
different regions of the EBS (including potential for influx from the
GOA) should be pursued.

Many studies have found inconclusive evidence for genetic population
structure in walleye pollock. Knowledge of stock structure is
particularly important for this species, given its commercial
importance. Therefore, a large scale study using the highest resolution
genetic tools available is recommended. Such a study would incorporate
samples throughout the range of walleye pollock, including North
America, Japan, and Russia, if possible. Data from thousands of SNP loci
should be screened, using next generation sequencing.


\address{%
James Ianelli\\
AFSC\\
line 1\\ line 2\\
}
\href{mailto:author1@work}{\nolinkurl{author1@work}}

\address{%
Author Two\\
Affiliation\\
line 1\\ line 2\\
}
\href{mailto:author2@work}{\nolinkurl{author2@work}}

